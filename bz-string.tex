%%
%%  making TOC (pdf bookmarks)
%%

%% typeset by lualatex
\documentclass{ltjsbook}

\usepackage{mktoc}

%% PDF のメタデータの設定
\hypersetup{
  % pdfusetitle, % derive the values for pdftitle and pdfauthor from \title and \author
  pdftitle={初級講座 弦理論 発展編},
  pdfauthor={B.ツヴィーバッハ},
  % pdfsubject={},
  % pdfcreator={},
  % pdfproducer={},
  pdfkeywords={超弦理論},
  bookmarksopen, % open up bookmark tree
}

%% 読み込むPDFの名前(相対パス)を入力。
\pdfName{src/string-1.pdf}

%% 表紙からページ番号が始まる(front)までのページ数。
%% 大文字ローマ数字でナンバリング。
\noPages{6}

%% front部分のページ数。
%% 小文字ローマ数字でナンバリング。
\frontPages{6}

%% mainページの開始番号(デフォルトは1)
\pageStartNum{327}

%% 何ページ目まで読み込むかを指定。
%% 全ページ読み込む場合は、コメントアウトする。
% \finalPage{30}

\begin{document}

%% PDFを読み込む
\makePages

%% --- 目次 (pdf bookmarks) の付加 ---

%% \noPages の部分 (Roman)
\bookmark[level=1,page=1]{表紙(カバー)}
\bookmark[level=1,page=7]{表紙}

%% front (roman)
\frontbookmark{1}{3}{目次}

%% main (arabic)
\mainbookmark{0}{327}{第15章 D-プレインとゲージ場}
\mainbookmark{0}{353}{第16章 弦のチャージと電荷}
\mainbookmark{0}{373}{第17章 閉弦のT双対性}
\mainbookmark{0}{397}{第18章 開弦およびD-プレインのT双対性}
\mainbookmark{0}{413}{第19章 電磁場を持つD-プレインとT双対性}
\mainbookmark{0}{431}{第20章 Born-Infeld理論とD-プレインの電磁場}
\mainbookmark{0}{449}{第21章 弦理論と素粒子物理}
\mainbookmark{0}{497}{第22章 弦の熱力学とブラックホール}
\mainbookmark{0}{527}{第23章 強い相互作用とAdS/CFT対応}
\mainbookmark{0}{571}{第24章 弦の共変な量子化}
\mainbookmark{0}{595}{第25章 弦の基本的な相互作用とRiemann面}
\mainbookmark{0}{635}{第26章 弦のダイヤグラムの構造とループ振幅}
\mainbookmark{0}{665}{参考文献について}
\mainbookmark{1}{668}{文献リスト}
\mainbookmark{0}{673}{索引}
\mainbookmark{0}{677}{奥付}


% \mainbookmark{0}{}{}
% \mainbookmark{1}{}{}

\end{document}
%% EOF